\chapter{Abstract}

The goal of this thesis is to provide a fast rendering technique to render translucent materials. The method we developed has low memory requirements, does not require a mesh UV mapping and requires little or not pre-processing. The method is well suited for real-time rendering applications such as computer games and digital interactive visualization. In addition, we will employ new BSSRDF analytical model that considers the directionality of the incoming light into account. This additional parameter introduces some interesting challenges that are not easily dealt with by previous approaches. 

The result is built by sampling other points on the surface using a special sampling pattern based on the optical properties of the material. Our method incrementally builds the result over a certain number of frames, rendering the model from different directions and storing it in a texture. The texture is then sampled using shadow mapping in order to obtain the final rendering. 

Using this approach, we obtained real-time results of 30 FPS for complex models of the magnitude normally employed in the computer game industry ($10^4$ triangles). The results we generate are close in appearance to a path traced solution. Our method then provides a fast and robust way to account for the direction of the incoming light in the computation, providing a more realistic results that the ones reachable with previous analytical models. 