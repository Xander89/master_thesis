\chapter{Conclusions}
\label{chap:conclusions}

In this thesis, we have presented an approch to rendering translucent materials using a new directional subsurface scattering BSSRDF models. We focused on creating a method that renders translucent materials in real-time, thus approximating well path traced results.

In chapter \ref{chap:previous} we gave an overview of the different approaches to rendering subsurface scattering in literature, identifying the gap in the knowledge in rendering translucent materials efficiently using a directional approach.

In chapter \ref{chap:theory}, we instroduced the mathematical concepts that are needed to give a basic understanding of light transport theory and BSSRDF models, introducing the nthe formulation of two BSSRDF dipole models, the standard dipole model and the directional dipole model. Using this knowledge, we gave the reader a theoretical introduction to our method, as well as scattering parameters and how they are acquired in chapter \ref{chap:method}. Chapter \ref{chap:method} provided an essential bridge between theory and implementation, that we described in chapter \ref{chap:implementation}. In the implementation, we described a method that employs the advantages of the formulation of the directional dipole in order to create a robust method that implements the directional dipole taking advantage of the GPU rendering pipeline.

In chapter \ref{chap:results}, we compared our result to path traced solutions, proving that our method can produce results that approximate well path-traced solutions, providing speed ups of four order of magnitude compared to path tracing on CPU. We also proved that our method for models of the size commonly used in the computer game industry performs in real-time on a high-end modern GPU, limiting the size of memory used to a minimum. 

Finally, in chapter	\ref{chap:futurework}, we introduced some possible ideas to expand our method in order to produce a higher quality result retaining its real-time capabilities.

To sum up, we think that most of the goals of the thesis stated in the introduction chapter have been satisfied. This thesis hopefully provides an insight on a new way to approach the efficient rendering of translucent materials using directional subsurface scattering. 